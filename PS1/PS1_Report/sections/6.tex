\documentclass[../report.tex]{subfiles}
\begin{document}
    
    \begin{frame}[t]
        \frametitle{6a. Discussion}
        \begin{normalsize}
            \begin{itemize}
                \setlength\itemsep{1em}\fontsize{6pt}{6pt}
                
                \item[]{\textbf{Between all color channels, which channel, in your opinion, most resembles a gray-scale conversion of the original.  Why do you think this?  Does it matter for each respective image? (For this problem, you will have to read a bit on how the eye works/cameras to discover which channel is more prevalent and widely used)} }
                
                \item[]{\selectfont\textcolor{blue}{i. I have used anaalyze_channels.py to analyze the channels of the image. It turned out that the green channel most resembles grayscale conversion with highest correlation ($0.9128$). \\
ii. I found that this is because the green channel is more prevalent and widely used in the eye along with camera sensors. \\
iii. I am not sure if this phenomenon differs for other images, but at least the images I have tested, the result was the same.
}}
                
            \end{itemize}
        \end{normalsize}
    \end{frame}

    \begin{frame}[t]
        \frametitle{6b. Discussion}
        \begin{normalsize}
            \begin{itemize}
                \setlength\itemsep{1em}\fontsize{6pt}{6pt}
                
                \item[]\textbf{What does it mean when an image has negative pixel values stored?  Why is it important to maintain negative pixel values?}
                
                \item[]{\selectfont\textcolor{blue}{Negative pixel values indicate normalized image representations (e.g., -1 to 1 range with 0 being the gray) or result from mathematical operations like image subtraction. In these cases, negative values have meanings that should not be simply ignored.
Clipping negatives to zero would lose important information about gradients, edges, and relationships between pixels. Clipped values cannot be recovered after all. Negative values might originate from mishandling the data, but one should examine it carefully to understand the exact cause.}}
                
            \end{itemize}
        \end{normalsize}
    \end{frame}

    \begin{frame}[t]
        \frametitle{6c. Discussion}
        \begin{normalsize}
            \begin{itemize}
                \setlength\itemsep{1em}\fontsize{6pt}{6pt}
                
                \item[]\textbf{In question 5, noise was added to the green channel and also to the blue channel. Which looks better to you? Why? What sigma was used to detect any discernible difference?}
                
                \item[]{\selectfont\textcolor{blue}{i. The green channel (ps1-5-a-1) looked much more noisier.\\
ii. I believe this has to do with the result we have derived in 6a. The green channel is used the most by the eye and camera sensors. So adding noise to the green channel will be more noticeable.\\
iii. sigma = 25 was used. The green channel showed noise even with smaller sigma, but for the blue channel it started to show comparably visible noise with sigma = 100. The latter still seemed to affect the background the most.}}
                
            \end{itemize}
        \end{normalsize}
    \end{frame}
    
\end{document}